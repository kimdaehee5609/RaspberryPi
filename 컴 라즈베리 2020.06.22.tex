%	-------------------------------------------------------------------------------
% 
%		2020년 6월 16일 첫작업
%
%
%
%
%
%
%
%	-------------------------------------------------------------------------------

%	\documentclass[12pt, a3paper, oneside]{book}
	\documentclass[12pt, a4paper, oneside]{book}
%	\documentclass[12pt, a4paper, landscape, oneside]{book}

		% --------------------------------- 페이지 스타일 지정
		\usepackage{geometry}
%		\geometry{landscape=true	}
		\geometry{top 			=10em}
		\geometry{bottom			=10em}
		\geometry{left			=8em}
		\geometry{right			=8em}
		\geometry{headheight		=4em} % 머리말 설치 높이
		\geometry{headsep		=2em} % 머리말의 본문과의 띠우기 크기
		\geometry{footskip		=4em} % 꼬리말의 본문과의 띠우기 크기
% 		\geometry{showframe}
	
%		paperwidth 	= left + width + right (1)
%		paperheight 	= top + height + bottom (2)
%		width 		= textwidth (+ marginparsep + marginparwidth) (3)
%		height 		= textheight (+ headheight + headsep + footskip) (4)



		%	===================================================================
		%	package
		%	===================================================================
%			\usepackage[hangul]{kotex}				% 한글 사용
			\usepackage{kotex}					% 한글 사용
			\usepackage[unicode]{hyperref}			% 한글 하이퍼링크 사용

		% ------------------------------ 수학 수식
			\usepackage{amssymb,amsfonts,amsmath}	% 수학 수식 사용
			\usepackage{mathtools}				% amsmath 확장판

			\usepackage{scrextend}				% 
		

		% ------------------------------ LIST
			\usepackage{enumerate}			%
			\usepackage{enumitem}			%
			\usepackage{tablists}				%	수학문제의 보기 등을 표현하는데 사용
										%	tabenum


		% ------------------------------ table 
			\usepackage{longtable}			%
			\usepackage{tabularx}			%
			\usepackage{tabu}				%




		% ------------------------------ 
			\usepackage{setspace}			%
			\usepackage{booktabs}		% table
			\usepackage{color}			%
			\usepackage{multirow}			%
			\usepackage{boxedminipage}	% 미니 페이지
			\usepackage[pdftex]{graphicx}	% 그림 사용
			\usepackage[final]{pdfpages}		% pdf 사용
			\usepackage{framed}			% pdf 사용

			
			\usepackage{fix-cm}	
			\usepackage[english]{babel}
	
		%	=======================================================================================
		% 	tikz package
		% 	
		% 	--------------------------------- 	
			\usepackage{tikz}%
			\usetikzlibrary{arrows,positioning,shapes}
			\usetikzlibrary{mindmap}			
			

		% --------------------------------- 	page
			\usepackage{afterpage}		% 다음페이지가 나온면 어떻게 하라는 명령 정의 패키지
%			\usepackage{fullpage}			% 잘못 사용하면 다 흐트러짐 주의해서 사용
%			\usepackage{pdflscape}		% 
			\usepackage{lscape}			%	 


			\usepackage{blindtext}
	
		% --------------------------------- font 사용
			\usepackage{pifont}				%
			\usepackage{textcomp}
			\usepackage{gensymb}
			\usepackage{marvosym}



		% Package --------------------------------- 

			\usepackage{tablists}				%


		% Package --------------------------------- 
			\usepackage[framemethod=TikZ]{mdframed}				% md framed package
			\usepackage{smartdiagram}								% smart diagram package



		% Package ---------------------------------    연습문제 

			\usepackage{exsheets}				%

			\SetupExSheets{solution/print=true}
			\SetupExSheets{question/type=exam}
			\SetupExSheets[points]{name=point,name-plural=points}


		% --------------------------------- 페이지 스타일 지정

		\usepackage[Sonny]		{fncychap}

			\makeatletter
			\ChNameVar	{\Large\bf}
			\ChNumVar	{\Huge\bf}
			\ChTitleVar		{\Large\bf}
			\ChRuleWidth	{0.5pt}
			\makeatother

%		\usepackage[Lenny]		{fncychap}
%		\usepackage[Glenn]		{fncychap}
%		\usepackage[Conny]		{fncychap}
%		\usepackage[Rejne]		{fncychap}
%		\usepackage[Bjarne]	{fncychap}
%		\usepackage[Bjornstrup]{fncychap}

		\usepackage{fancyhdr}
		\pagestyle{fancy}
		\fancyhead{} % clear all fields
		\fancyhead[LO]{\footnotesize \leftmark}
		\fancyhead[RE]{\footnotesize \leftmark}
		\fancyfoot{} % clear all fields
		\fancyfoot[LE,RO]{\large \thepage}
		%\fancyfoot[CO,CE]{\empty}
		\renewcommand{\headrulewidth}{1.0pt}
		\renewcommand{\footrulewidth}{0.4pt}
	
	
	
		%	--------------------------------------------------------------------------------------- 
		% 	tritlesec package
		% 	
		% 	
		% 	------------------------------------------------------------------ section 스타일 지정
	
			\usepackage{titlesec}
		
		% 	----------------------------------------------------------------- section 글자 모양 설정
			\titleformat*{\section}					{\large\bfseries}
			\titleformat*{\subsection}				{\normalsize\bfseries}
			\titleformat*{\subsubsection}			{\normalsize\bfseries}
			\titleformat*{\paragraph}				{\normalsize\bfseries}
			\titleformat*{\subparagraph}				{\normalsize\bfseries}
	
		% 	----------------------------------------------------------------- section 번호 설정
			\renewcommand{\thepart}				{\arabic{part}.}
			\renewcommand{\thesection}				{\arabic{section}.}
			\renewcommand{\thesubsection}			{\thesection\arabic{subsection}.}
			\renewcommand{\thesubsubsection}		{\thesubsection\arabic{subsubsection}}
			\renewcommand\theparagraph 			{$\blacksquare$ \hspace{3pt}}

		% 	----------------------------------------------------------------- section 페이지 나누기 설정
			\let\stdsection\section
			\renewcommand\section{\newpage\stdsection}



		%	--------------------------------------------------------------------------------------- 
		% 	\titlespacing*{commandi} {left} {before-sep} {after-sep} [right-sep]		
		% 	left
		%	before-sep		:  수직 전 간격
		% 	after-sep	 	:  수직으로 후 간격
		%	right-sep

			\titlespacing*{\section} 			{0pt}{1.0em}{1.0em}
			\titlespacing*{\subsection}	  		{0ex}{1.0em}{1.0em}
			\titlespacing*{\subsubsection}		{0ex}{1.0em}{1.0em}
			\titlespacing*{\paragraph}			{0em}{1.5em}{1.0em}
			\titlespacing*{\subparagraph}		{4em}{1.0em}{1.0em}
	
		%	\titlespacing*{\section} 			{0pt}{0.0\baselineskip}{0.0\baselineskip}
		%	\titlespacing*{\subsection}	  		{0ex}{0.0\baselineskip}{0.0\baselineskip}
		%	\titlespacing*{\subsubsection}		{6ex}{0.0\baselineskip}{0.0\baselineskip}
		%	\titlespacing*{\paragraph}			{6pt}{0.0\baselineskip}{0.0\baselineskip}
	

		% --------------------------------- recommend		섹션별 페이지 상단 여백
		\newcommand{\SectionMargin}				{\newpage  \null \vskip 2cm}
		\newcommand{\SubSectionMargin}			{\newpage  \null \vskip 2cm}
		\newcommand{\SubSubSectionMargin}		{\newpage  \null \vskip 2cm}


		%	--------------------------------------------------------------------------------------- 
		% 	toc 설정  - table of contents
		% 	
		% 	
		% 	----------------------------------------------------------------  문서 기본 사항 설정
			\setcounter{secnumdepth}{4} 		% 문단 번호 깊이
			\setcounter{tocdepth}{2} 			% 문단 번호 깊이 - 목차 출력시 출력 범위

			\setlength{\parindent}{0cm} 		% 문서 들여 쓰기를 하지 않는다.


		%	--------------------------------------------------------------------------------------- 
		% 	mini toc 설정
		% 	
		% 	
		% 	--------------------------------------------------------- 장의 목차  minitoc package
			\usepackage{minitoc}

			\setcounter{minitocdepth}{1}    	%  Show until subsubsections in minitoc
%			\setlength{\mtcindent}{12pt} 	% default 24pt
			\setlength{\mtcindent}{24pt} 	% default 24pt

		% 	--------------------------------------------------------- part toc
		%	\setcounter{parttocdepth}{2} 	%  default
			\setcounter{parttocdepth}{0}
		%	\setlength{\ptcindent}{0em}		%  default  목차 내용 들여 쓰기
			\setlength{\ptcindent}{0em}         


		% 	--------------------------------------------------------- section toc

			\renewcommand{\ptcfont}{\normalsize\rm} 		%  default
			\renewcommand{\ptcCfont}{\normalsize\bf} 	%  default
			\renewcommand{\ptcSfont}{\normalsize\rm} 	%  default


		%	=======================================================================================
		% 	tocloft package
		% 	
		% 	------------------------------------------ 목차의 목차 번호와 목차 사이의 간격 조정
			\usepackage{tocloft}

		% 	------------------------------------------ 목차의 내어쓰기 즉 왼쪽 마진 설정
			\setlength{\cftsecindent}{2em}			%  section

		% 	------------------------------------------ 목차의 목차 번호와 목차 사이의 간격 조정
			\setlength{\cftsecnumwidth}{2em}		%  section





		%	=======================================================================================
		% 	flowchart  package
		% 	
		% 	------------------------------------------ 목차의 목차 번호와 목차 사이의 간격 조정
			\usepackage{flowchart}
			\usetikzlibrary{arrows}


		%	=======================================================================================
		% 		makeindex package
		% 	
		% 	------------------------------------------ 목차의 목차 번호와 목차 사이의 간격 조정
%			\usepackage{makeindex}
%			\usepackage{makeidy}


		%	=======================================================================================
		% 		각주와 미주
		% 	

		\usepackage{endnotes} %미주 사용


		%	=======================================================================================
		% 	줄 간격 설정
		% 	
		% 	
		% 	--------------------------------- 	줄간격 설정
			\doublespace
%			\onehalfspace
%			\singlespace
		
		

	% 	============================================================================== itemi Global setting

	
		%	-------------------------------------------------------------------------------
		%		Vertical spacing
		%	-------------------------------------------------------------------------------
			\setlist[itemize]{topsep=0.0em}			% 상단의 여유치
			\setlist[itemize]{partopsep=0.0em}			% 
			\setlist[itemize]{parsep=0.0em}			% 
%			\setlist[itemize]{itemsep=0.0em}			% 
			\setlist[itemize]{noitemsep}				% 
			
		%	-------------------------------------------------------------------------------
		%		Horizontal spacing
		%	-------------------------------------------------------------------------------
			\setlist[itemize]{labelwidth=1em}			%  라벨의 표시 폭
			\setlist[itemize]{leftmargin=8em}			%  본문 까지의 왼쪽 여백  - 4em
			\setlist[itemize]{labelsep=3em} 			%  본문에서 라벨까지의 거리 -  3em
			\setlist[itemize]{rightmargin=0em}			% 오른쪽 여백  - 4em
			\setlist[itemize]{itemindent=0em} 			% 점 내민 거리 label sep 과 같은면 점위치 까지 내민다
			\setlist[itemize]{listparindent=3em}		% 본문 드려쓰기 간격
	
	
			\setlist[itemize]{ topsep=0.0em, 			%  상단의 여유치
						partopsep=0.0em, 		%  
						parsep=0.0em, 
						itemsep=0.0em, 
						labelwidth=1em, 
						leftmargin=2.5em,
						labelsep=2em,			%  본문에서 라벨 까지의 거리
						rightmargin=0em,		% 오른쪽 여백  - 4em
						itemindent=0em, 		% 점 내민 거리 label sep 과 같은면 점위치 까지 내민다
						listparindent=0em}		% 본문 드려쓰기 간격
	
%			\begin{itemize}
	
		%	-------------------------------------------------------------------------------
		%		Label
		%	-------------------------------------------------------------------------------
			\renewcommand{\labelitemi}{$\bullet$}
			\renewcommand{\labelitemii}{$\bullet$}
%			\renewcommand{\labelitemii}{$\cdot$}
			\renewcommand{\labelitemiii}{$\diamond$}
			\renewcommand{\labelitemiv}{$\ast$}		
	
%			\renewcommand{\labelitemi}{$\blacksquare$}   	% 사각형 - 찬것
%			\renewcommand\labelitemii{$\square$}		% 사각형 - 빈것	
			






% ------------------------------------------------------------------------------
% Begin document (Content goes below)
% ------------------------------------------------------------------------------
	\begin{document}
	
			\dominitoc
			\doparttoc			




			\title{라즈베리 파이}
			\author{김대희}
			\date{2020년 6월}
			\maketitle


			\tableofcontents 		% 목차 출력
%			\listoffigures 			% 그림 목차 출력
			\cleardoublepage
			\listoftables 			% 표 목차 출력





		\mdfdefinestyle	{con_specification} {
						outerlinewidth		=1pt			,%
						innerlinewidth		=2pt			,%
						outerlinecolor		=blue!70!black	,%
						innerlinecolor		=white 			,%
						roundcorner			=4pt			,%
						skipabove			=1em 			,%
						skipbelow			=1em 			,%
						leftmargin			=0em			,%
						rightmargin			=0em			,%
						innertopmargin		=2em 			,%
						innerbottommargin 	=2em 			,%
						innerleftmargin		=1em 			,%
						innerrightmargin		=1em 			,%
						backgroundcolor		=gray!4			,%
						frametitlerule		=true 			,%
						frametitlerulecolor	=white			,%
						frametitlebackgroundcolor=black		,%
						frametitleaboveskip=1em 			,%
						frametitlebelowskip=1em 			,%
						frametitlefontcolor=white 			,%
						}



%	================================================================== Part			라즈베리 파이
	\addtocontents{toc}{\protect\newpage}
	\part{라즈베리 파이}
	\noptcrule
	\parttoc				



%	================================================================== Part			라즈베리 파이
	\addtocontents{toc}{\protect\newpage}
	\chapter{라즈베리 파이}
	\noptcrule

	\newpage	
	\minitoc





% ----------------------------------------------------------------------------- 개요
%
% -----------------------------------------------------------------------------
	\section{라즈베라 파이 개요}


라즈베리 파이(영어: Raspberry Pi)는 영국 잉글랜드의 라즈베리 파이 재단이 학교와 개발도상국에서 기초 컴퓨터 과학의 교육을 증진시키기 위해 개발한 신용카드 크기의 싱글 보드 컴퓨터이다.[4][5][6][7][8] 초기의 라즈베리 파이는 엘레멘트14/프리미어 파넬, RS 콤포넌트와의 허가된 제조 협정을 통해 제작되었다.[9]

라즈베리 파이는 그래픽 성능이 뛰어나면서도 가격이 저렴(-VAT 제외- 1세대 모델 A와 1세대 모델 A+의 경우 25달러, 1세대와 2세대를 포함한 나머지 모델의 경우 35달러)한 것이 특징이다.

라즈베리 파이는 모두 동일한 비디오코어 IV GPU와,[10] 싱글코어 ARMv6에 호환되는 CPU 또는 신형의 ARMv7에 호환되는 쿼드코어(라즈베리 파이 2), 1 GB의 RAM(라즈베리 파이 2), 512 MB(라즈베리 파이 1 B와 B+),[11] 또는 256 MB(모델 A와 A+, 구형 모델 B)의 메모리를 포함한다. 이들은 SD 카드 슬롯 (모델 A 와 B) 또는 부팅 가능한 매체와 지속적인 정보 저장을 위한 마이크로SDHC를 갖추고 있다.[12]

라즈베리 파이 2를 제외한 라즈베리 파이 모델들은 브로드컴의 BCM2835 단일 칩 시스템을 사용하며,[13] 이 칩에는 ARM1176JZF-S 700 MHz 싱글코어프로세서(일반 데스크탑은 보통 2500 MHz~3500 MHz), 비디오코어 IV VGA[10] 와 512 메가바이트 RAM이 들어 있다. 그리고 라즈베리 파이의 프로세서는 오버클럭시 최대 1000 MHz까지의 성능을 발휘 할 수 있으며, 하드 디스크 드라이브나 솔리드 스테이트 드라이브를 내장하고 있지 않으며, SD 카드(B+,2 B+ 모델은 Micro SD Card를 사용)를 외부 기억장치로 사용한다. 새로 출시한 2 모델 B는 ARM Cortex-A7 0.9GHz프로세서와 램용량이 1GB로 성능이 업그레이드되어 출시 되었다.[12] 라즈베리 재단은 컴퓨터 교육 증진을 위해 2가지 모델을 내놓았으며, 각각 25달러와 35달러로 책정되었다. 2012년 2월 29일 재단에서 35달러짜리 모델의 주문을 받기 시작하였다.[14]

2014년, 라즈베리 파이 재단은 원래의 라즈베리 파이와 계산 능력이 같은 임베디드 시스템의 일부로 사용하기 위한 '계산 모듈'을 출시하였다. 
2015년 초, 차세대 라즈베리 파이인 라즈베리 파이 2가 출시되었다. 이 새로운 컴퓨터 보드는 처음에는 한 가지 형식(모델 B)이었으며, 쿼드 코어 ARM Cortex-A7 CPU와 1GB RAM에 나머지 사양은 모델 B+와 유사했다. 라즈베리 파이2는 모델 B와 같은 미화 35의 가격을 유지했으며,[17] 미화 20의 모델 A+는 여전히 판매되었다. 2015년 11월, 라즈베리 파이 재단은 그보다 작은 제품인 미화 5의 라즈베리 파이 제로를 출시하였다.

라즈베리 파이 재단은 데비안과 아치 리눅스 ARM 배포판의 다운로드를 제공하고, 주요 프로그래밍 언어로 파이썬의 사용을 촉진하며, BBC 베이직을 지원한다. C, C++, 자바, 펄, 루비, 스퀵 스몰토크 등의 언어가 사용 가능하다.

2015년 6월 8일 현재, 약 5~6백 만 대의 라즈베리 파이가 판매되었다.[22][23] 영국에서 가장 빠른 속도로 팔리고 있는 개인용 컴퓨터로, 라즈베리 파이는 800만 대가 판매된 워드 프로세서 '암스트래드 PCW'에 이어 두 번째로 많이 선적되었다.




% ----------------------------------------------------------------------------- 역사
%
% -----------------------------------------------------------------------------
	\section{ 역사}



목차
1	역사
% ----------------------------------------------------------------------------- 빌매전
	\section{ 빌매전}


빌매전

% ----------------------------------------------------------------------------- 판매
	\section{ 판매}

1.2	판매
1.3	2013년
1.4	2015년
1.5	2016년
1.6	2017년
1.7	2019년
1.8	2020년

% ----------------------------------------------------------------------------- 하드웨어
	\section{ 하드웨어}

2	하드웨어
2.1	명세
2.1.1	참고


% ----------------------------------------------------------------------------- 소프트웨어
	\section{ 소프트웨어}

3	소프트웨어
3.1	아키텍처
3.2	포트될 예정 혹은 포트된 리눅스 배포판 혹은 다른 운영 체제

% ----------------------------------------------------------------------------- 주변장치
	\section{ 주변장치}

4	주변장치


% ----------------------------------------------------------------------------- 평가
	\section{ 평가}

5	평가

% ----------------------------------------------------------------------------- 사용
	\section{ 사용}

6	사용


% ----------------------------------------------------------------------------- 같이보기
	\section{ 같이보기}

7	같이 보기

% ----------------------------------------------------------------------------- 참고문헌
	\section{ 참고문헌}
8	참고 문헌

% ----------------------------------------------------------------------------- 책
%
% -----------------------------------------------------------------------------
	\section{라즈베리 파이 책 }


\paragraph{라즈베리 파이 시작하기}

\begin{itemize}[					
		topsep=0.0em,			
		parsep=0.0em,			
		itemsep=0em,			
		leftmargin=	3	em,
		labelwidth=	1	em,			
		labelsep=		1	 em			
]					
	\item	제목 	: 
	\item	지은이 	:
	\item	옮김 	:
	\item	출판사 	: 
	\item	출판일 	:	 
	\item	도서관 	: 2020.06.19 수정 분관 004.16 22

\end{itemize}					




% ----------------------------------------------------------------------------- 외부링크
	\section{ 외부링크}

9	외부 링크














% ------------------------------------------------------------------------------
% End document
% ------------------------------------------------------------------------------
\end{document}


	\href{https://www.youtube.com/watch?v=SpqKCQZQBcc}{태양경배자세A}
	\href{https://www.youtube.com/watch?v=CL3czAIUDFY}{태양경배자세A}


https://docs.google.com/spreadsheets/d/1-wRuFU1OReWrtxkhaw9uh5mxouNYRP8YFgykMh2G_8c/edit#gid=0
+

https://seoyeongcokr-my.sharepoint.com/:f:/g/personal/02017_seoyoungeng_com/Ev8nnOI89D1LnYu90SGaVj0BTuckQ46vQe1HiVv-R4qeqQ?e=S3iAHi